\section*{Abstract}
\hfill\\
\textbf{Background:} This report looks at the possibility of diagnosing lung cancer using circulating miRNA. There has been a lot of research in this field, but little research from a machine learning perspective.
\\\\
\textbf{Motivation:} Using machine learning to diagnose lung cancer is practical as current methods for diagnosing lung cancer are resource-intensive and the tumor is typically found at a late stage when the survival rate is low.
\\\\
\textbf{Experiments:} I tried to collect all available datasets on circulating miRNA and lung cancer. Then I tried to find whether there were any patterns in case-control characteristics using different statistical tests. I have done machine learning internally in the different datasets and externally across multiple datasets. I also made a web application for visualizing the data in the different datasets.
\\\\
\textbf{Contributions:} The main contributions of this project are to make all available datasets on circulating miRNA and lung cancer into a common format so that the work can be built upon by other researchers, and a web application that can be used by researchers to visualize the data.
\\\\
\textbf{Results:} The result of this project is that I was not able to find any patterns in case-control characteristics that could replicate across datasets, with only a few exceptions. Furthermore, machine learning across different datasets was not able to learn any patterns in most cases, despite good results when using machine learning internally in a dataset. This suggests that findings in single studies rarely have external validity.

\iffalse

This paper provides a template for writing AI project rapports for either the AI specialisation project; masters "datateknikk" or masters "informatikk". The use of the template is recommended and is written in english as we encourage students to submit their project and masters theses in English. 
The template does not form a compulsory style that you are obliged to use. However, the format and contents are a result of a joint AI group initiative thus providing a common starting point for all AI students. For a given project tuning of the template may still be required. Such tuning might involve moving a chapter to a section or vice versa due to the nature of the project. 

The abstract is your sales pitch which encourages people to read your work but unlike sales it should be realistic with respect to the contributions of the work. It should include:
\begin{itemize}
\item the field of research
\item a brief motivation for the work
\item what the research topic is and
\item the research approach(es) applied. 
\item contributions
\end{itemize}

The abstract length should be roughly half a page of text --- without lists, tables or figures.  

\fi
