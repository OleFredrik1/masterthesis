% (C) Anders Kofod-Petersen
\documentclass[a4paper]{book}
\usepackage[english]{babel}						% Correct English hyphenation
%\usepackage[latin1]{inputenc}						% Allow for non-English letters
\usepackage[table,dvipsnames]{xcolor}
\usepackage{graphicx}							% To include graphics
\usepackage{natbib}								% Correct citations
\usepackage{fancyheadings}						% Nice header and footer
\usepackage[linktocpage,colorlinks]{hyperref}			% PDF hyperlink
\usepackage{geometry} 							% Better geometry
%\usepackage[center]					% For cropping documents
\usepackage{amsmath}
\usepackage{amsfonts}
\usepackage{pgfplots}

% B5 (uncomment to convert to B5 format)
\geometry{b5paper}
\setcounter{secnumdepth}{3}
% Author
% Fill in here, and use commands in the text. 
\newcommand{\thesisAuthor}{Ole Fredrik Borgundvåg Berg}
\newcommand{\thesisTitle}{Circulating miRNA and lung cancer: \\ - an analysis of available data}
\newcommand{\thesisType}{Master thesis}
\newcommand{\thesisDate}{Spring 2022}

% PDF info
\hypersetup{pdfauthor={\thesisAuthor}}
\hypersetup{pdftitle={\thesisTitle}}
\hypersetup{pdfsubject={\thesisType}}
\hypersetup{linkcolor=black}
\hypersetup{citecolor=black}
\hypersetup{urlcolor=black}

%Fancy headings
%\pagestyle{fancy}
%\pagestyle{fancyplain}
%\renewcommand{\chaptermark}[1]{\markboth{#1}{}}
%\renewcommand{\sectionmark}[1]{\markright{#1}{}}
%\lhead[\fancyplain{}{\thepage}]{\fancyplain{}{\let\uppercase\relax\leftmark}}
%\rhead[\fancyplain{}{\let\uppercase\relax\rightmark}]{\fancyplain{}{\thepage}}
%\chead[\fancyplain{}{}]{\fancyplain{}{}}
%\lfoot[\fancyplain{}{}]{\fancyplain{}{}}
%\cfoot[\fancyplain{}{}]{\fancyplain{}{}}
%\rfoot[\fancyplain{}{}]{\fancyplain{}{}}

% Citation format
\bibliographystyle{apalike}
\bibpunct{[}{]}{;}{a}{,}{,}



\usepackage{rotating}
\usepackage{csvsimple}
\usepackage{caption}
\usepackage{subcaption}
\usepackage{tablefootnote}
\usepackage{todonotes}
\usepackage{tikz}
\usepackage{ifthen}
\usepackage{arrayjob}
\usepackage{pgfmath}
\usepackage{pgfplotstable}
\usepackage{pythontex}
\usepackage{siunitx}
\usepackage{tikz-network}
\usepackage{adjustbox}
\usepackage{svg}
\usepackage{makecell}
\usepackage{blindtext} %This package generates automatic text
\usepackage{epigraph} 

\usepgfplotslibrary{statistics}

\extrafloats{300}

\usepackage{alphalph}
\renewcommand*{\thesubfigure}{%
\alphalph{\value{subfigure}}%
}%


\usepackage{xparse}% http://ctan.org/pkg/xparse
\usepackage{etoolbox}% http://ctan.org/pkg/etoolbox
\newcounter{listtotal}\newcounter{listcntr}%
\NewDocumentCommand{\studiesarr}{o}{%
  \setcounter{listtotal}{0}\setcounter{listcntr}{-1}%
  \renewcommand*{\do}[1]{\stepcounter{listtotal}}%
  \expandafter\docsvlist\expandafter{\studies}%
  \IfNoValueTF{#1}
    {\studies}% \names
    {% \names[<index>]
     \renewcommand*{\do}[1]{\stepcounter{listcntr}\ifnum\value{listcntr}=#1\relax##1\fi}%
     \expandafter\docsvlist\expandafter{\studies}}%
}

\newcommand{\Autoref}[1]{%
  \begingroup%
  \def\chapterautorefname{Chapter}%
  \def\sectionautorefname{Section}%
  \def\subsectionautorefname{Subsection}%
  \autoref{#1}%
  \endgroup%
}

\definecolor{caribbeangreen}{rgb}{0.8, 1, 0.8}
\definecolor{candypink}{rgb}{1, 0.8, 0.8}
\definecolor{mygray}{rgb}{0.99, 0.99, 0.99}
\newcommand{\mynum}[1]{
    \pgfmathparse{#1 < 0.01}
    \ifthenelse{\pgfmathresult > 0}{
        \num[round-mode=figures, exponent-mode=scientific]{#1}
    }{
        \num{#1}
    }
}
\pgfset{
  foreach/parallel foreach/.style args={#1in#2via#3}{evaluate=#3 as #1 using {{#2}[#3-1]}},
}
\newcommand\studies{Abdollahi2019, Asakura2020, Bianchi2011, Boeri2011, Chen2019, Duan2021,
           Fehlmann2020, Halvorsen2016, Jin2017, Keller2009,
           Keller2014, Keller2020, Kryczka2021, Leidinger2011, Leidinger2014,
           Leidinger2016, Li2017, Marzi2016, Nigita2018,
           Patnaik2012, Patnaik2017, Qu2017, Reis2020, Wozniak2015,
           Yao2019, Zaporozhchenko2018}
\newcommand\studylen{26}

\begin{pycode}
import os
import sys
import pandas as pd
import numpy as np
from scipy.stats import ttest_ind
sys.path.append(os.getcwd())
studies = ["Abdollahi2019", "Asakura2020", "Bianchi2011", "Boeri2011", "Chen2019", "Duan2021",
           "Fehlmann2020", "Halvorsen2016", "Jin2017", "Keller2009",
           "Keller2014", "Keller2020", "Kryczka2021", "Leidinger2011", "Leidinger2014",
           "Leidinger2016", "Li2017", "Marzi2016", "Nigita2018",
           "Patnaik2012", "Patnaik2017", "Qu2017", "Reis2020", "Wozniak2015",
           "Yao2019", "Zaporozhchenko2018"]
\end{pycode}

