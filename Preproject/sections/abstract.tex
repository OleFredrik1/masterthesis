\section*{Abstract}

There has been many studies on the diagnostic value of using circulating microRNA to diagnose lung cancer, which is an important research area, as an earlier diagnosis of lung cancer will save many lives. 
While there has been multiple studies on this research areas, few have tried to combine the results from multiple studies, and none have tried to collect all available datasets, which was the goal of this project.
In this project I have collected all available datasets in order to compare them and to look at the results from using one dataset to diagnose lung cancer in another dataset.

The contributions has been that all datasets are collected made into a common format, which would make it easier to do research on this area in the future. Another contribution is the overview of the availability of datasets on the subject, and the properties of the available datasets. In short, the datasets from most studies were not given upon request, which leads to no possibility of replication and limits the possibility for collecting many datasets in order to make research on a larger combined datasets which could have given more statistical power and better estimates.

Finally, trying to diagnose across datasets generally led to quite poor results with low diagnostic value, which suggest that the results from the different studies often don't replicate. However, these results came from very naïve machine learning, and more research are needed in order to see if more advanced machine learning can find patterns across datasets better.
\iffalse

This paper provides a template for writing AI project rapports for either the AI specialisation project; masters "datateknikk" or masters "informatikk". The use of the template is recommended and is written in english as we encourage students to submit their project and masters theses in English. 
The template does not form a compulsory style that you are obliged to use. However, the format and contents are a result of a joint AI group initiative thus providing a common starting point for all AI students. For a given project tuning of the template may still be required. Such tuning might involve moving a chapter to a section or vice versa due to the nature of the project. 

The abstract is your sales pitch which encourages people to read your work but unlike sales it should be realistic with respect to the contributions of the work. It should include:
\begin{itemize}
\item the field of research
\item a brief motivation for the work
\item what the research topic is and
\item the research approach(es) applied. 
\item contributions
\end{itemize}

The abstract length should be roughly half a page of text --- without lists, tables or figures.  

\fi
